In order to explain why the proximity of generated samples to the anchor is relevant to the efficiency during training, 
one can consider a simple example in Euclidean space.
Imagine images as input to a \ac{nn}, which projects them onto $f_{\theta}(x) \in \mathbb{R}^d$, where $theta$ are the parameters of the \ac{nn}.
The effect of the distance between the anchor and the positive (negative) sample on the loss is visualized in \autoref{fig:hard_easy_samples_dist_effect_loss}.


\begin{figure}[htbp]
    \centering
    \includesvg[width=300pt]{images/Hard_easy_samples_dist_effect_loss}
    \caption{The impact of the distance between a generated sample and its anchor on the loss function.
    Hard samples convey more (gradient) information than easy samples and thus, have a higher loss value.
    While distant positive pairs are considered hard, for negative samples, small proximity ones are considered hard.}
    \label{fig:hard_easy_samples_dist_effect_loss}
\end{figure}