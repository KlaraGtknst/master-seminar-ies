\begin{abstract}
    %The abstract should briefly summarize the contents of the paper in
    %150--250 words.
    % unsupervised learning
    Unsupervised learning techniques are of interest to many researchers, as they allow training models on data without any labels.
    % SSL
    \acl{ssl} is a subset of unsupervised learning, where labels are generated from unlabelled training data.
    % CL
    One application of \acl{ssl} is \acl{cl}, which is a technique that is frequently applied in representation learning.
    % representation learning
    In this context, the representations of similar samples are supposed to be encoded within close range of each other, 
    while the representations of dissimilar samples are pushed apart.
    % instance recognition
    To this end, pre-text tasks such as instance recognition are used: 
    Each sample is considered its own class and other samples or transformations of the sample are classified in a binary manner as dis-/similar.
    % dis-/similar pairs
    \acl{cl} is used to train the model on pairs of similar and dissimilar samples.
    The selection of these dis-/similar pairs is of particular interest 
    since difficult pairs cause great learning opportunities for the model. 
    While transformations of a sample, i.e. the so-called anchor, are considered similar, 
    the selection of dissimilar pairs is more challenging.
    % this paper
    This paper reviews different approaches for finding sample pairs in \acl{cl}, with a focus on hard sample mining.
    
    \keywords{\acl{cl}  \and \acl{ssl} \and Hard sample mining.}
\end{abstract}