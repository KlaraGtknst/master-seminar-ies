\subsection{Mining on manifolds}\label{subsec:mining_manifolds}

\citet{mining_manifolds_2018} combines different definitions of proximity to mine 
for hard positives and negatives.


\begin{figure}[h]%
    \centering
    \subfloat[\centering Euclidean $k$ nearest neighbour $NN^e_k$ (orange).]
    {{\includegraphics[width=5cm]{images/euclidean_NN.png} }}%
    \qquad
    \subfloat[\centering Manifold $k$ nearest neighbour $NN^m_k$ (purple).]
    {{\includegraphics[width=5cm]{images/manifold_NN.png} }}%
    \qquad
    \subfloat[\centering Hard positives $NN^m_k \textbackslash NN^e_k$(green).]
    {{\includegraphics[width=5cm]{images/hard_positives_manifold.png} }}%
    \qquad
    \subfloat[\centering Hard negatives $NN^e_k \textbackslash NN^m_k$(red).]
    {{\includegraphics[width=5cm]{images/hard_negatives_euclidean.png} }}%

    \caption{Visualization of different proximity definitions and 
    the hard negatives/positives from \citet{mining_manifolds_2018}.
    The anchor is the black point.}%
    \label{fig:mining_manifolds_vis}%
\end{figure}