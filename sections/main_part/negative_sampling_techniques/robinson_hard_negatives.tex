\subsection{Robinson's Hard Negatives}\label{subsec:robinson_hard_negatives}

Since positive pairs ($x$, $x^+$) are considered to originate from the same class, we denote $\rho(c), c \in \mathcal{C}$ as the distribution over the latent classes.
Moreover, let $h: \mathcal{X} \rightarrow \mathcal{C}$ be the ground truth assigning class labels $c \in \mathcal{C}$ to inputs $x \in \mathcal{X}$.
Hence, $x \sim x'$ if $h(x) = h(x')$.
Assuming that $\rho(c)$ is uniformly distributed, and that $x^- \sim q = p$ where the anchor $x$ 
is drawn from the data distribution $p$, \citet{robinson_contrastive_2021} 
propose a novel approach to sample hard negatives from $q \neq p$.

This method balances both the risk of sampling \ac{fn} and the degree of hardness of the samples obtained.
Moreover, via the concentration parameter $\beta$, the user can decide how hard the negatives should be and thus, how likely they have the same target as the anchor.
Large values of $\beta$ lead to sampling very hard negative samples.
The authors of \citet{robinson_contrastive_2021} propose negative samples from $q^-_{\beta}(x^-)$, 
which enforces $x$, $x^-$ having different classes by conditioning on dissimilar classes.
By computing a scalar product of both representations $f(x)$, $f(x^-)$ in \eqref{eq:q_beta}, 
they encourage similar represented samples.
Unfortunately, it is unclear how to sample from $q^-_{\beta}(x^-)$ and thus, 
the authors propose a way to rewrite the equation to facilitate sampling.

\begin{align} 
    q^-_{\beta}(x^-) = q_\beta(x^-|h(x) \neq h(x^-)) \propto \exp(\beta f(x)^\text{T}f(x^-))\cdot p(x^-) 
\label{eq:q_beta}
\end{align} 


Since sampling could both yield \ac{fn} and \ac{tn}, it is possible to rewrite \eqref{eq:q_beta_minus} 
displaying both cases in \eqref{eq:fn_tn}.
The first term corresponds to the probability of sampling \ac{tn}, i.e. $h(x) \neq h(x')$,
 while the second term corresponds to the probability of sampling \ac{fn} with $x \sim x'$.

\begin{align}
    q_\beta(x^-) &= \rho(c)q^{-}_\beta(x^-) + (1-\rho(c))q^{+}_\beta(x^-)
    \label{eq:fn_tn}
\end{align}

The distribution $q^{+}_\beta(x^-)$ can described in more detail.
Given that $x$ and $x^-$ are from the same class, 
the probability of sampling $x^-$ is proportional to the exponential of the scalar product of the 
representations of $x$ and $x^-$ as described in \eqref{eq:q_beta_plus}.

\begin{align}
    q^{+}_\beta(x^-) &= q_\beta(x^-|h(x) = h(x^-)) &\propto \exp(\beta f(x)^\text{T}f(x^-))\cdot p^+(x^-) 
    \label{eq:q_beta_plus}
\end{align}

Finally, we can rewrite \eqref{eq:fn_tn} to obtain the desired distribution $q^-_{\beta}(x^-)$ as shown in \eqref{eq:q_beta_minus}.
This version of the distribution allows for sampling from $q^-_{\beta}(x^-)$, since we can sample from $q_\beta(x^-) \approx p$ 
and we can approximate $q^{+}_\beta(x^-)$ by sampling from a set of transformations.    % rejection sampling & Monte Carlo sampling

\begin{align}
    q^{-}_\beta(x^-) &= \frac{(q_\beta(x^-)-(1-\rho(c))q^{+}_\beta(x^-))}{\rho(c)} 
    \label{eq:q_beta_minus}
\end{align}



% \begin{align*}
%     q^{-}_\beta(x^-) &=  q_\beta(x^-|h(x) \neq h(x^-)) 
% &\propto \exp(\beta f(x)^\text{T}f(x^-))\cdot p(x^-) 
% \\&= \rho(c)q^{-}_\beta(x^-) + (1-\rho(c))q^{+}_\beta(x^-)
% \\&= \rho(c)\frac{(q_\beta(x^-)-(1-\rho(c))q^{+}_\beta(x^-))}{\rho(c)} + (1-\rho(c))q_\beta(x^-|h(x) = h(x^-)) &\propto \rho(c)q^{-}_\beta(x^-) + \exp(\beta f(x)^\text{T}f(x^-))\cdot p^+(x^-) 
% \end{align*}