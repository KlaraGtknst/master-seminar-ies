\section{Critic}\label{sec:critic}

Even though the presented sampling techniques are very promising, after gathering and evaluating the information, 
some critical points arise which will be discussed in the following.

% positive augmentations
Many scientists have led their paper with the statement that the generation of positive samples has already been subject to many studies.
However, the majority of the papers subject to this work used random transformations to generate positive samples \citep{robinson_contrastive_2021,adversarial_2020,swav_2020}.
While the effort invested in initially identifying effective augmentation strategies for positive samples is evident, 
the consideration given to positive sample generation in these papers appears relatively superficial.

% explainable sample generation
From an explainability perspective, 
using cluster-based sample generation techniques appears to be a more promising approach 
than simply stating that a randomly selected augmentation from a set of transformations was applied, 
without providing further motivation or detail.

% manifold mining
\citet{mining_manifolds_2018}'s manifold mining approach, on the other hand, poses an interesting and more graspable approach to sample generation. 
However, according to the paper, the manifold is calculated only once at the beginning of the training process.
This is possible either due to the assumption that the manifold does not change during the training process or 
due to computational costs.
Since the manifold is calculated based on the Euclidean nearest neighbour graph which is dependent on the embedding, 
the manifold structure could change during the training process of the embedding function.
Therefore, recalculation of the manifold could be of interest.

% clustering-based approaches
Some clustering-based approaches such as \ac{pcl} \citep{PCL_2021} seem to need $M$ clusterings per iteration 
which seems computationally expensive. 
However, since a batch of samples changes the embedding, recalculation of the clusters is necessary.
Nevertheless, the computational costs of this approach should be considered when evaluating \ac{pcl}'s overall usability.

% cosine similarity
Even though most papers use the dot product to calculate the similarity between samples, 
\citet{mining_potential_2024} and \ac{la} \citep{local_aggr_2019} use cosine distance 
without providing further justification. % do authors say why they use it? LA: No, Potential: No
% LA uses cosine similarity for background samples, it is unclear weather it is used for close neighbours.
The cosine similarity is a measure of the angle between two vectors and does not consider the magnitude.
Therefore, the use of cosine similarity to measure the similarity between samples appears questionable.


% high dimensional spaces
The curse of dimensionality states that in high-dimensional spaces, distances become meaningless.
Consequently, cluster-based approaches in high-dimensional spaces might not be as effective as in lower-dimensional spaces.
Since dimensionality reduction always leads to information loss, 
one has to consider this drawback when using clustering-based approaches 
such as \ac{drc} \citep{DRC_2020}, \ac{swav} \citep{swav_2020}, \ac{pcl} \citep{PCL_2021}, 
\ac{la} \citep{local_aggr_2019} or mining manifolds \citep{mining_manifolds_2018}.
To avoid information loss, i.e. working on the original high-dimensional data, 
the usage of approaches such as \ac{mochi} \citep{mochi_2020} seems beneficial.

% DRC & L_{ap}
The clustering-based approach \ac{drc} \citep{DRC_2020} was briefly discussed in \autoref{subsec:SampleViaClustering}.
\ac{drc} aims to address the issue of existing methods enforcing the representations of samples 
and their augmentations to be assigned to the same cluster regardless of their \ac{af}.
Hence, they consider both the \ac{af} and the \ac{ap} during clustering.
Since the \ac{ap} is calculated via the softmax function of the \ac{af}, 
it is debatable whether the usage of \ac{ap} in the loss is necessary.